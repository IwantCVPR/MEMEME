\documentclass[10pt,conference,a4paper]{IEEEtran}
% *** CITATION PACKAGES ***
%
\ifCLASSOPTIONcompsoc
% IEEE Computer Society needs nocompress option
% requires cite.sty v4.0 or later (November 2003)
\usepackage[nocompress]{cite}
\else
% normal IEEE
\usepackage{cite}
\fi

% *** GRAPHICS RELATED PACKAGES ***
%
\usepackage{subfigure} % add by kk
\usepackage{float} % add by kk
\usepackage{CJK} % add by kk
\usepackage{amsmath} % add by kk
\usepackage{amssymb} % add by kk
\usepackage{longtable} % add by kk
\usepackage{multirow} % add by kk
\usepackage{array} % add by kk
\usepackage{chngpage} %  add by kk
\usepackage{hyperref} % add by kk
\usepackage{mathrsfs} % add by kk

\ifCLASSINFOpdf
\usepackage[pdftex]{graphicx}
% declare the path(s) where your graphic files are
\graphicspath{{../pdf/}{../jpeg/}}
% and their extensions so you won't have to specify these with
% every instance of \includegraphics
\DeclareGraphicsExtensions{.pdf,.jpeg,.png}
\else
% or other class option (dvipsone, dvipdf, if not using dvips). graphicx
% will default to the driver specified in the system graphics.cfg if no
% driver is specified.
\usepackage[dvips]{graphicx}
% declare the path(s) where your graphic files are
\graphicspath{{../eps/}}
% and their extensions so you won't have to specify these with
% every instance of \includegraphics
\DeclareGraphicsExtensions{.eps}
\fi
% correct bad hyphenation here
\hyphenation{op-tical net-works semi-conduc-tor}



\begin{document}
	\title{Automatic Mathematical Expression Alignment and Recognition System}
	% author names and affiliations
	% use a multiple column layout for up to three different
	% affiliations
	\author{
		\IEEEauthorblockN{TIP$^{1}$}
		\IEEEauthorblockA{
			$^{1}$ National Laboratory of Pattern Recognition, \\
			Institute of Automation, Chinese Academy of Sciences, Beijing, China \\
			Email: \{TIP\}@nlpr.ia.ac.cn \\
		}
	}
	%\footnote{National Laboratory of Pattern Recognition, Institute of Automation, Chinese Academy of Sciences}
	% make the title area
	\maketitle
	% As a general rule, do not put math, special symbols or citations
	% in the abstract
	\begin{abstract}
		Most previous mathematical expression recognition systems are built by rule-based methods. In this paper, we proposed a novel method to learn to adjust distorted mathematical expressions and output structural result in \LaTeX\ form ... 
	\end{abstract}
	
	\IEEEpeerreviewmaketitle
	\section{Introduction}
	Mathematical Expression (ME) plays a role in scientific area and automatic ME recognition has been studied for a long time. A typical ME recognition system usually contains three parts: mathematical symbol segmentation, mathematical symbol recognition and structural analysis.
	recognition has been studied for a long time. 
	Review ME recognition methods and point out the problems ...
		
	Review CNN ...
		
	In this paper, our contribution mainly falls into two aspects: 1.Proposed an automatic alignment and recognition system; 2. Build a new benchmark composed of distorted ME images.
		
	The remainder of this paper begins with ... We then ... We then report the results ...
		
	\section{Related Work}
	...
	
	\section{Proposed Method}
	% Before giving the network structure, we should give a block diagram of ME image processing, with the CNN as a block in it. Explain every step in the processing flow.
	% The overview part needs substantial extension. It should give a complete view of the method to the readers.
	\subsection{Overview}
	...
	Our system mainly contains three parts: a spatial transformer network for ME alignment, a fully convolution network for MS detection and recognition and a tree based structural analysis module.
	\subsection{Automatic alignment}
	We adopt some ideas in Spatial Transformer Network to perform the ME alignment. In [stn], the network takes the category label, which can be regarded as a $1 \times 1$ matrix, as the supervised information for each image and the transformation parameters are learned in an unsupervised way. While considering that the afterwards fully convolution network for MS detection and recognition takes a $n \times n$ matrix as the supervised message and the label containing spatial information is set on condition that the ME has already been properly aligned, it is essential to learn the transformation in a supervised way and pretrain the alignment network.
	
	The STN uses a $3 \times 2$ transformation matrix to represent translation, rotation, scale and shear. One natural way to directly learn the $3 \times 2$ matrix, and if we only consider translation and rotation, the transformation matrix $T$ can be represented as 
	\begin{equation}
		T = 
		\begin{bmatrix}
		\cos\theta & \sin\theta\\ 
		-\sin\theta & \cos\theta\\
		x & y
		\end{bmatrix}
	\end{equation}
	However, there are two problems need to be dealt with. The first one is that dimensions of $x, y, \sin(\theta), \cos(\theta)$ are not unified and the loss may be small enough for translation parameters but far from ideal for rotation ones. The second problem is that the network is required to regress complicated nonlinear functions like $\sin, \cos$ or compound ones, and it is not an easy task. Consequently, we split $T$ into three matrices $T_1, T_2, T_3$ each of which represents translation, rotation and scale, respectively.
	\begin{equation}
		T_1 = 
		\begin{bmatrix}
		1 & 0 & 0\\
		0 & 1 & 0\\
		x & y & 1
		\end{bmatrix}
	\end{equation}
	\begin{equation}
		T_2 = 
		\begin{bmatrix}
		\cos\theta & \sin\theta & 0\\
		-\sin\theta & \cos\theta & 0\\
		0 & 0 & 1
		\end{bmatrix}
	\end{equation}
	\begin{equation}
	T_3 = 
	\begin{bmatrix}
	s & 0 & 0\\
	0 & s & 0\\
	0 & 0 & 1
	\end{bmatrix}
	\end{equation}
	During the training stage, we first unify the dimension of $x, y, \theta, s$ and regress the unified transformation parameters by a multi-task structure shown in Fig.[Alignment Network]. And during the test stage, we first integrate $T_1, T_2, T_3$ into $T$ by
	\begin{equation}
		T = T_2 \cdot T_3 \cdot T_1
	\end{equation}
	and then remove the last column of $T$. The rest operation is the same as $Grid Generator$ and $Sampler$ in [stn].
		
	\subsection{Mathematical symbol detection and recognition}	
    ...
	\subsection{Joint training}
	Differentiable of alignment network
	\subsection{Structural analysis}
	...
	\section{Experiments}
	...
	\subsection{Implementation details}
	...
	\subsection{Databases}
	...
	\subsubsection{Infty}
	...
	\subsubsection{Self-based databases}
	...
	\subsection{Effectiveness of split transformation matrix}
	...
	\subsection{Effectiveness of joint-training}
	No joint-training
	Joint-training:
		1. Continue supervising transformation
		2. Training transformation in unsupervised way
	
	
	
	\section{Conclusion}
	...
	
	%\bibliographystyle{IEEEtran}
	%\bibliography{ref}
\end{document}


